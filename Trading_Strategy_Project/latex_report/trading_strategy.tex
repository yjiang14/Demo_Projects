\documentclass{amia}
\usepackage{graphicx}
\usepackage[labelfont=bf]{caption}
\usepackage[superscript,nomove]{cite}
\usepackage{color}
\usepackage{amssymb}
\usepackage{hyperref}
\usepackage{multirow}
\usepackage{subfigure}
\usepackage{amsmath}
\usepackage{hyperref}
\usepackage{url}


\geometry{left=1.5cm, top=0.8cm, right=1.5cm, bottom=1.0cm, footskip=1.5cm} % Configure page margins with geometry

\hypersetup{
    colorlinks=true,
    linkcolor=blue,
    filecolor=magenta,      
    urlcolor=cyan,
    citecolor = black
}
\begin{document}

\title{Trading Strategy Project}

\author{Yan Jiang}

%\author{Yan Jiang, Degrees$^{1}$}
\institutes{
%    $^1$Institution, City, State, Country (if applicable); $^2$Institution, City, State, Country (if applicable)\\
}

\maketitle

\textit{This is code part: \href{https://github.com/yjiang14/Demo_Projects/tree/master/Trading_Strategy_Project/code}{Git Repo Link (https://github.com/yjiang14/Demo\_Projects/tree/master/Trading\_Strategy\_Project/code)}}

\section*{Goal}
This project focuses on building trading strategies for JPM to maximize the portfolio value in two-year window. 

\section*{Introduction}
The project contains two parts:
\vspace{-3mm}
\begin{itemize}
	\item Trained JPM data starting with \$100,000 cash using manual strategy and strategy learner (Q-Learning) to generate best tradings strategy in in-sample (2008/1/1 - 2009/12/31) and tested it in out-of-sample (2010/1/1 - 2011/12/31). There only were three allowable positions: 1000 shares long, 1000 shares short and 0 shares
	\item Evaluated those two strategies by calculating daily portfolio value based on strategy-generated tradings. Benchmark was also created to compare with our strategies. Benchmark also started with \$100,000, investigated 1000 shares on the first trading day and held that position. 
	\end{itemize}
\vspace{-3mm}
This report will interpret strategy building processes and display a chart to evaluate those two strategies. The code and related files can be found through this link: \textit{\href{https://github.com/yjiang14/Demo_Projects/tree/master/Trading_Strategy_Project}{Git Repo Link (https://github.com/yjiang14/Demo\_Projects/tree/master/Trading\_Strategy\_Project)}}


\subsection*{Strategy Building}
In this project, Strength Index (RSI), Bollinger Bands B\% (BBP) and Commodity Channel Index (CCI) were selected to build manual strategy and strategy learner (Q-Learning). 
In manual strategy, BBP, RSI, and CCI were calculated and different values were assigned to each of them at their different conditions as below:
\vspace{-2mm}
\begin{figure}[H]
	\centering
	\includegraphics[height=1.5cm]{pics/ms_val.png}
	%\captionsetup{justification = centering}
	%\caption{Benchmark vs. Manual Strategy vs. Strategy Learner in Portfolio Values (In-sample period)}
	%\label{fig:figure4}
\end{figure}
\vspace{-5mm}
Those criterions I selected in this study are usually applied in stocking analytics as described in section. Those values were summed up as each day indication and made long/short/hold decision based on this indication value:
\vspace{-2mm}
\begin{figure}[H]
	\centering
	\includegraphics[height=1.8cm]{pics/ms_formula.png}
	%\captionsetup{justification = centering}
	%\caption{Benchmark vs. Manual Strategy vs. Strategy Learner in Portfolio Values (In-sample period)}
	%\label{fig:figure4}
\end{figure}
\vspace{-5mm}
In strategy learner, Q-learning was used to frame this trading strategy. Combined stock trading processes with project requirements, three components were defined: (1) actions: (i)long 1000 shares, (ii) short 1000 shares, (iii) holding; (2) states: discretized RSI in to 8 intervals, BBP into 10 intervals, CCI into 8 intervals, and permutated all intervals. Therefore, there are 8*8*10=640 states. Those states were assigned to each day according to daily RSI, BBP, and CCI intervals combination; (3) rewards: daily return. In the Q-learning, for each day (day loop): (1) calculates indicators and then get a state; (2) request action based on this state by querying current Q table (find a action with maximum rewards) (3) calculate reward based requested action at this state and update reward based on step (3) reward and step (2) state in Q table; repeat above day loop multiple times until cumulative return stops improving. 

\subsection*{Strategy Evaluation}
In order to evaluate manual strategy and strategy learner, impact of 0.005, commission of \$9.95, and start value of \$100,000 were utilized, and the daily portfolio values were normalized for benchmark, manual strategy and strategy learner.The Figure 1 presents normalized portfolio values for benchmark, manual strategy and strategy learner. 
\vspace{-1mm}
\begin{figure}[H]
	\centering
	\includegraphics[height=4cm]{pics/exp1.png}
	%\captionsetup{justification = centering}
	\caption{Benchmark vs. Manual Strategy vs. Strategy Learner in Portfolio Values}
\end{figure}

\end{document}

